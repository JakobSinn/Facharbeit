\documentclass[11pt, a4paper, headings=standardclasses]{scrartcl}
\usepackage[T1]{fontenc}
\usepackage[utf8]{inputenc}
\usepackage[top=2.5cm, bottom=2.5cm, left=2.5cm, right=3.5cm]{geometry}
\usepackage[style=authortitle-ibid, autocite=footnote, backend=biber, dashed=false]{biblatex}
\usepackage{url}
\usepackage[hidelinks]{hyperref}
\bibliography{Facharbeit}
\begin{document}
\renewcommand{\baselinestretch}{1.5}
\subject{Facharbeit}
\author{Jakob Sinn}
\title{Influence of the trade between \\ Asia and Europe on India}
\maketitle
\thispagestyle{empty}
\clearpage
\tableofcontents
\pagenumbering{Roman}
\clearpage
\section{Introduction}
\pagenumbering{arabic}
\subsection{Summary}

\subsection{Thesis Statement}
My thesis stament: the trade between Asia and Europe, in both directions, had a profound effect on India by introducing Islam, influencing the political structure, and forming the basis for colonisation.
\subsection{Methodology}
 
\section{Trade up to the age of Islam}
Trade in this time, before the trade movements were consolidated under the peace and security offered by the islamic conquests of Arabia, moved both on the land route via the middle east and from India over the india ocean to the Red Sea, and from there on to Egypt or the eastern coast of the Medditeranean.\autocite[Chapter 7]{Rome} For both of these routes, the Medditeranean served as the distributor to Europe.\footnote{These routes are recorded in the first--century BC anonymous work \emph{Periplus of the Erythraean Sea}.}

\subsection{Traded goods}
While most of the goods traded were spice and other special, often medical, plants, luxuries like silk and gems were also increasingly in demand from a prosperous upper class in the Roman empire.\autocite{RIS}
\subsubsection{Spice}
At the time, spices were appreciated not just as enhancers of food, but primarily as medicine\autocite{MST}. The spices came from India, modern-day Indonesia, and China. Cinnamon, for example, is native to Sri Lanka, Myanmar, Bangladesh, and the Malabar Coast of India itself\autocite{Cinnamon}.
\subsubsection{Precious stones}
Some Gems could only be sourced outside Europe, and due to their compactness were attractive trade goods. However, as the demand for spice far outpaced that for percious stones\autocite{Rome}, they were not the main focus of many merchants.

In this era, the stones were often engraved, a technique found in almost all early cultures. The craftsmen performing this work in Greece and Rome were so famous that many of thier names have survived to this day. Precious stones were given as state gifts and donated to temples.\autocite{RG}
\subsubsection{Silk}
Silk had not yet reached the prominence it would in the late middle ages, but was still a well-known import from northern China. There are recorded complaints about the desire of Roman wives for silk depleting the silver and gold reserves, demonstrating the trade deficit between west and east.\autocite[Chapter 13]{Rome}
\subsection{Trade routes}
In this age, India was mostly a distribution hub for the spices and other luxury goods from Indonesia and China. Reports of trading fleets exchanging goods in Indian harbours even made it all the way to ancient Greece and Rome, where they were written down and are still preserved today, for example in the writings of Ptolomy.\autocite[p.~148-150]{Rome}

\subsubsection{To India from Indonesia}

Spices were mainly originating from the islands of Indonesia and Malaya, and, over their history, many empires and kingdoms thrived of controlling the trade routes between the isles.\autocite{Indonesia}

There was also a direct trade between Indonesia and western Africa. This led to the colonisation of Madagascar by the Austronesians.\autocite{Madagascar}

The stories of their \emph{Raft-men}, who came to India on rudderless rafts to sell the spices and return home with traded goods, traveled along the trading routes to Europe. They are mentioned, for example, by Pliny.\autocite[Chapter 8]{Rome}

\subsubsection{From China}

From China, there were three main routes.

The northernmost, known as the \emph{Scythian route}, went from the north of the yellow river, next to the central silk road. In modern-day northwest China, it seperated, and made its way north of the Aral and Caspian Sea, passing through Volgograd and ending on the mouth of the river Don on the Black sea, the ancient Greek colony of Tana\"{i}s. This, of course, opened the way to Byzantium via the Black Sea.\autocite[149]{Rome}

Most of the non-spices exported from China, primarily silk, were carried along the \emph{Silk Road}. The route tended to start in northern China, bypassing Tibet, the Himalayas, ans thus India, to the north. From there, it led through the Parthian empire\autocite{SilkRome} to Medditeranen ports like Antioch.\autocite[Chapter 7, Maps 2 and 3]{Rome}

Many spices were imported into the southern ports of China like Canton, from where spices and other goods were also exported by sea to modern-day Sri Lanka. Furthermore, goods were also shipped overland from southern China to India.\autocite[Maps 4 and 5]{Rome} The importance of this route must not be underastimated; Chinese records describe Roman trade missions in cities in modern-day northern Vietnam\autocite{curtin_1984}, and Ptolemy tells of a place called \emph{Cattigara}, identified today as \emph{\'{O}c Eo}, an archeological site in the Mekong River delta where Roman gods from the second century CE have been excavated.\autocite{OcEo}

\subsubsection{India as a trade nexus}
India, while producing many of the goods traded, also served as a trading centre of the Asia-Europe trade. This was due to its geographical advantages with acces to Arabia and the Red Sea to the west, as well as the Indian ocean and land connections to China to the east.

In general, trade flowed mostly eastwards, while the recipients in Europe mostly paid in gold coin. Indias position as a trading centre, especially after the emergence of the Roman empire, is demontrated by find of Roman-struck gold coinage from the imperial age.\autocite[100]{curtin_1984}

\subsubsection{From India to Europe}

From India, the main route was via the Red Sea. This route was on the upswing after the pacification under Augustus, coinciding with the coins mentioned earlier.\autocite[Chapter 7]{Rome}

The overland route crossed the Parthian Empire, wich acted as an intermeadiary between the Chinese trade missions and the Roman traders. They often obscured the origins of their goods to prevent being bypassed.\autocite{SilkRome}

\subsubsection{Development under the Romans}

During the Roman empire, both diplomacy and war opened up new routes. Ships from Europe even made it to southern India, bringing home a new spice: black pepper. It was even added to meals, remaining the only spice used this way in Antiquity.\autocite{RIS}

This made spice trade even more important in relation to other imports.\autocite{SilkRome, Rome} These efforts also made Alexandria, then part of the Roman empire, one of the foremost trading cities in the world.\autocite{SpiceTrade}
\section{Networks of Trade up to the Age of Discovery}

\subsection{Changes in Arabia and the middle east}

\subsubsection{Beginnings of the Arabic empire}
After the first conquests under Muhammad, the Islamic or Arabian empire was created, stretching from Spain to India\autocite[Section \textit{Achievements}]{Umayyad}. This made the land routes to India and China much more attractive. This era formed the golden age of the silk road, with increasing contact between Europe and China.

\subsubsection{The Abbasid caliphate}

After the removal of the previous widely unpopular\autocite{Umayyad} Umayyad caliphate, the Islamic world was ruled by the nominally Shia \emph{Abbasid Caliphate}. The beginnig of Abbasid rule in 750 CE was followed by the inauguration of the \emph{House of Wisdom} in Bagdad, marking beginning the Islamic golden age.\autocite{Abbasid}

In this time, the caliphate was shaken by civil wars and seperatist dynasties, including remnants of the Umayyads.\autocite[Section \textit{End}]{Umayyad} Nonetheless, this centralised power made travel and trade easier along the overland Silk Road.

\subsubsection{Cultural influences on trade}

Muhammad himself was a merchant\autocite{Muhammad}, and Islam was, in the first centuries of its spread, very much a trade-based religion. Even though arab muslims conquered vast lands, these campaingns were not at all motivated by religion\autocite[200]{Lapidus}, and active conversion efforts were rare. Instead, many converted to access the Islamic trade networks.\autocite{SilkRoadIslam} In this environment, merchants, especially arabs, were able to significantly expand their trade networks.

\subsubsection{Changing trade routes}


Due to the islamization of the land routes between the Medditeranean, these land routes became more attractive for Chinese trade missions. However, Christians were reluctant trade via the Levant, embroiled in conflict by the crusades, to the benefit of Alexandria.\autocite[Paragraph 5]{UN}

\subsection{India}

In the context of this research report, India is defined not stricly along the geographical lines used today, but as a culturally hegemenous area that included modern-day India, Pakistan, Bangladesh, and the north of Myanmar.

While India was mostly an exporter of ideas during antiquity (Buddhism reached China in this period\autocite{SilkRoadIslam}), the trading began having a noticable influence on Indi by introducing new ideas and, most importantly, Islam as a new religion.

\subsubsection{Political structure during the middle ages}

Between the fall of the Gupta around 450 CE\autocite[Section \textit{Gupta}]{India} and the next significant centralized authority, the islamic Delhi sultanate, in 1206\autocite{Delhi}, India was ruled mostly by small kingdoms and principalities. These states often exercised informal control over their neighbours, and many smaller kingoms paid tribute.\autocite{India}

\subsubsection{Cultural transfers from the west}

After the 8th century, muslim traders began to bring their religion with them.\autocite{SilkRoadIslam} Many along the trade routes eagerly converted, and to this day, muslims make up a majority of Pakistanis and a significant minority of Indians.\autocite{India}

Scientists in India were also exposed to new ideas from the centres of research in the islamic world, especially in medicine.\autocite{SilkRoadIslam}

\subsubsection{Spice production}

Spice production remained a major economic factor in India. An example is Cardamom, native to Malabar in south-western India.\autocite[71]{Rome}

\subsection{Europe}

Between the fall of the western Roman empire and the first excursions by merchants, the spice trade remained an important part of the economy.\autocite{Yale} This was, especially in the middle ages, driven by the high demand for these spices in Europe. 

According to Freedman, ``Spices were ubiquitous in medieval gastronomy''.\autocite[3]{MST} Spice usage expanded signifcantly in comparison to antiquity, but medicinal use remained important.\autocite{RIS}

\subsubsection{Venice}

Up until the exploitation of routes around Africa to access the spices more directly, Venice controlled the trade in the Medditeranean sea. Venitian merchants sold the spice to distributors in Europe, having gained their monopoly in the Chioggia war (1387-81) with Genoa.\autocite{SpiceTrade}
\section{Trading under colonial rule and up to Indian independence}
\section{Effects on India}
\section{Conclusion}
\clearpage
\appendix
\printbibliography

\end{document}