\documentclass[11pt, a4paper, headings=standardclasses]{scrartcl}
\usepackage[T1]{fontenc}
\usepackage[utf8]{inputenc}
\usepackage[top=2.5cm, bottom=2.5cm, left=2.5cm, right=3.5cm]{geometry}
\usepackage[style=authortitle-ticomp, autocite=footnote, backend=biber, dashed=false]{biblatex}
\usepackage{url}
\usepackage[hidelinks]{hyperref}
\bibliography{Facharbeit}
\begin{document}
\renewcommand{\baselinestretch}{1.5}
\subject{Facharbeit}
\author{Jakob Sinn}
\title{Influence of the trade between \\ Asia and Europe on India}
\maketitle
\thispagestyle{empty}
\clearpage
\tableofcontents
\pagenumbering{Roman}
\clearpage
\section{Introduction}
\pagenumbering{arabic}
\subsection{Summary}

\subsection{Thesis Statement}
My thesis stament: the trade between Asia and Europe, in both directions, had a profound effect on India by introducing Islam, influencing the political structure, and forming the basis for colonisation.
\subsection{Methodology}
 
\section{Trade up to the age of Islam}
Trade in this time, before the trade movements were consolidated under the peace and security offered by the islamic conquests of Arabia, moved both on the land route via the middle east and from India over the india ocean to the Red Sea, and from there on to Egypt or the eastern coast of the Medditeranean.\autocite[Chapter 7]{Rome} For both of these routes, the Medditeranean served as the distributor to Europe.\footnote{These routes are recorded in the first--century BC anonymous work \emph{Periplus of the Erythraean Sea}.}

\subsection{Traded goods}
While most of the goods traded were spice and other special, often medical, plants, luxuries like silk and gems were also increasingly in demand from a prosperous upper class in the Roman empire.\autocite{RIS}
\subsubsection{Spice}
At the time, spices were appreciated not just as enhancers of food, but primarily as medicine\autocite{MST}. The spices came from India, modern-day Indonesia, and China. Cinnamon, for example, is native to Sri Lanka, Myanmar, Bangladesh, and the Malabar Coast of India itself\autocite{Cinnamon}.
\subsubsection{Precious stones}
Some Gems could only be sourced outside Europe, and due to their compactness were attractive trade goods. However, as the demand for spice far outpaced that for percious stones\autocite{Rome}, they were not the main focus of many merchants.

In this era, the stones were often engraved, a technique found in almost all early cultures. The craftsmen performing this work in Greece and Rome were so famous that many of thier names have survived to this day. Precious stones were given as state gifts and donated to temples.\autocite{RG}
\subsubsection{Silk}
Silk had not yet reached the prominence it would in the late middle ages, but was still a well-known import from northern China. There are recorded complaints about the desire of Roman wives for silk depleting the silver and gold reserves, demonstrating the trade deficit between west and east.\autocite[Chapter 13]{Rome}
\subsection{Trade routes}
In this age, India was mostly a distribution hub for the spices and other luxury goods from Indonesia and China. Reports of trading fleets exchanging goods in Indian harbours even made it all the way to ancient Greece and Rome, where they were written down and are still preserved today, for example in the writings of Ptolomy.\autocite[p.~148-150]{Rome}

\subsubsection{To India from Indonesia}

Spices were mainly originating from the islands of Indonesia and Malaya, and, over their history, many empires and kingdoms thrived of controlling the trade routes between the isles.\autocite{Indonesia}

There was also a direct trade between Indonesia and western Africa. This led to the colonisation of Madagascar by the Austronesians.\autocite{Madagascar}

The stories of their \emph{Raft-men}, who came to India on rudderless rafts to sell the spices and return home with traded goods, traveled along the trading routes to Europe. They are mentioned, for example, by Pliny.\autocite[Chapter 8]{Rome}

\subsubsection{From China}

From China, there were three main routes.

The northernmost, known as the \emph{Scythian route}, went from the north of the yellow river, next to the central silk road. In modern-day northwest China, it seperated, and made its way north of the Aral and Caspian Sea, passing through Volgograd and ending on the mouth of the river Don on the Black sea, the ancient Greek colony of Tana\"{i}s. This, of course, opened the way to Byzantium via the Black Sea.\autocite[149]{Rome}

Most of the non-spices exported from China, primarily silk, were carried along the \emph{Silk Road}. The route tended to start in northern China, bypassing Tibet, the Himalayas, ans thus India, to the north. From there, it led through the Parthian empire\autocite{SilkRome} to Medditeranen ports like Antioch.\autocite[Chapter 7, Maps 2 and 3]{Rome}

Many spices were imported into the southern ports of China like Canton, from where spices and other goods were also exported by sea to modern-day Sri Lanka. Furthermore, goods were also shipped overland from southern China to India.\autocite[Maps 4 and 5]{Rome} The importance of this route must not be underastimated; Chinese records describe Roman trade missions in cities in modern-day northern Vietnam\autocite{curtin_1984}, and Ptolemy tells of a place called \emph{Cattigara}, identified today as \emph{\'{O}c Eo}, an archeological site in the Mekong River delta where Roman gods from the second century CE have been excavated.\autocite{OcEo}

\subsubsection{India as a trade nexus}
India, while producing many of the goods traded, also served as a trading centre of the Asia-Europe trade. This was due to its geographical advantages with acces to Arabia and the Red Sea to the west, as well as the Indian ocean and land connections to China to the east.

In general, trade flowed mostly eastwards, while the recipients in Europe mostly paid in gold coin. Indias position as a trading centre, especially after the emergence of the Roman empire, is demontrated by find of Roman-struck gold coinage from the imperial age.\autocite[100]{curtin_1984}

\subsubsection{From India to Europe}

From India, the main route was via the Red Sea. This route was on the upswing after the pacification under Augustus, coinciding with the coins mentioned earlier.\autocite[Chapter 7]{Rome}

The overland route crossed the Parthian Empire, wich acted as an intermeadiary between the Chinese trade missions and the Roman traders. They often obscured the origins of their goods to prevent being bypassed.\autocite{SilkRome}

\subsubsection{Development under the Romans}

During the Roman empire, both diplomacy and war opened up new routes. Ships from Europe even made it to southern India, bringing home a new spice: black pepper. It was even added to meals, remaining the only spice used this way in Antiquity.\autocite{RIS}

This made spice trade even more important in relation to other imports.\autocite{SilkRome, Rome} These efforts also made Alexandria, then part of the Roman empire, one of the foremost trading cities in the world.\autocite{SpiceTrade}
\section{Networks of Trade up to the Age of Discovery}

\subsection{Changes in Arabia and the middle east}

\subsubsection{Beginnings of the Arabic empire}
After the first conquests under Muhammad, the Islamic or Arabian empire was created, stretching from Spain to India\autocite[Section \textit{Achievements}]{Umayyad}. This made the land routes to India and China much more attractive. This era formed the golden age of the silk road, with increasing contact between Europe and China.

\subsubsection{The Abbasid caliphate}

After the removal of the previous widely unpopular\autocite{Umayyad} Umayyad caliphate, the Islamic world was ruled by the nominally Shia \emph{Abbasid Caliphate}. The beginnig of Abbasid rule in 750 CE was followed by the inauguration of the \emph{House of Wisdom} in Bagdad, marking beginning the Islamic golden age.\autocite{Abbasid}

In this time, the caliphate was shaken by civil wars and seperatist dynasties, including remnants of the Umayyads.\autocite[Section \textit{End}]{Umayyad} Nonetheless, this centralised power made travel and trade easier along the overland Silk Road.

\subsubsection{Cultural influences on trade}

Muhammad himself was a merchant\autocite{Muhammad}, and Islam was, in the first centuries of its spread, very much a trade-based religion. Even though arab muslims conquered vast lands, these campaingns were not at all motivated by religion\autocite[200]{Lapidus}, and active conversion efforts were rare. Instead, many converted to access the Islamic trade networks.\autocite{SilkRoadIslam} In this environment, merchants, especially arabs, were able to significantly expand their trade networks.

\subsubsection{Changing trade routes}

Due to the islamization of the land routes between the Medditeranean, these land routes became more attractive for Chinese trade missions. However, Christians were reluctant trade via the Levant, embroiled in conflict by the crusades, to the benefit of Alexandria.\autocite[Paragraph 5]{UN}

\subsection{India}

In the context of this research report, India is defined not stricly along the geographical lines used today, but as a culturally hegemenous area that included modern-day India, Pakistan, Bangladesh, and the north of Myanmar.

While India was mostly an exporter of ideas during antiquity (Buddhism reached China in this period\autocite{SilkRoadIslam}), the trading began having a noticable influence on Indi by introducing new ideas and, most importantly, Islam as a new religion.

\subsubsection{Political structure during the middle ages}

Between the fall of the Gupta around 450 CE\autocite[Section \textit{Gupta}]{India} and the next significant centralized authority, the islamic Delhi sultanate, in 1206\autocite{Delhi}, India was ruled mostly by small kingdoms and principalities. These states often exercised informal control over their neighbours, and many smaller kingoms paid tribute.\autocite{India}

\subsubsection{Cultural transfers from the west}

After the 8th century, muslim traders began to bring their religion with them.\autocite{SilkRoadIslam} Many along the trade routes eagerly converted, and to this day, muslims make up a majority of Pakistanis and a significant minority of Indians.\autocite{India}

Scientists in India were also exposed to new ideas from the centres of research in the islamic world, especially in medicine.\autocite{SilkRoadIslam}

\subsubsection{Spice production}

Spice production remained a major economic factor in India. An example is Cardamom, native to Malabar in south-western India.\autocite[71]{Rome}

\subsection{Europe}

Between the fall of the western Roman empire and the first excursions by merchants, the spice trade remained an important part of the economy.\autocite{Yale} This was, especially in the middle ages, driven by the high demand for these spices in Europe. 

According to Freedman, ``Spices were ubiquitous in medieval gastronomy''.\autocite[3]{MST} Spice usage expanded signifcantly in comparison to antiquity, but medicinal use remained important.\autocite{RIS}

\subsubsection{Venice}

Up until the exploitation of routes around Africa to access the spices more directly, Venice controlled the trade in the Medditeranean sea. Venitian merchants sold the spice to distributors in Europe, having gained their monopoly in the Chioggia war (1387-81) with Genoa.\autocite{SpiceTrade}

\section{From the discovery of a seaway to India to trading companies}

After 1498, the trade routes discussed earlier were bypassed by rounding Africa. The focus was not political, but mostly commercial; in da Gamas second voyage to India, enough profit was made to finance a third.\autocite[Section \textit{The second voyage}]{Vasco}

Soon, however, colonial powers like the Portuguese and the Dutch established permanent settlements and later colonies, followed by large \emph{companies} and English and French colonizers.

\subsection{Portuguese expeditions and colonies}

In 1498, a Portuguese expedition under the command on \emph{Vasco da Gama} reachend India by rounding Africa, completely bypassing the traditional trade routes\autocite{VdG}. This expedition made a lasting and deep impact on the trade between Asia and Europe, ending the dominance of the Islamic world and Venice as waystations and distributors of the spice trade.\autocite{GLO}

\subsubsection{The first voyage}

The expeditions were financed by King Manuel I, who had ascended to the throne in 1495. While court politcs were cetainly a main factor in starting the expedition, the the explanation for why da Gama was chosen to lead it, the underlying goals were to break the muslim and venitian monopoly on trade with the east.\autocite{Vasco}

The expedition established a \emph{factory}, or trading post, in Calicut (known today as \textit{Kozhikode}). After a conflict with arab merchants in the city, the factory is destroyed by a mob and  about 50 portuguese are massacred, while anchored portuguese ships are unable to help.\autocite{1550}

\subsubsection{The second voyage}

Da Gamas second voyage was intended as a trading mission and a punitive expedition to target Calicut, in response to the massacre of the factory.\autocite{Vasco}

The fleet massacred hundreds of pilgrims, women and children after capturing a muslim fleet conducting Hajj. He justified this act of piracy as retaliation for the Calicut massacre,\autocite{1550} but his actions would set the tone for Indo-European relations in the next centuries. This massacre, and many later conflicts, are attributable to religious hatred by the colonizers against muslims\autocite[382]{FT}

\subsubsection{The \textit{Estado da India}}

The first portuguese fort on India soil was constructed in 1503 in the domain of the Raja of Cochin, an ally of the protuguese.\autocite[383]{FT} The true foundation to the colony, however, was laid in the capture of the island of Goa in 1510. Aimed at fully controlling the spice trade and aided by the insufficient naval power avaliable to the indian authorities, this maritime Empire aquired the name \textit{Estado da India}. It captured Malacca in 1511, and Ormuz in the Persian Gulf in 1515, ensuring supremacy on the seas for a century.\autocite[382--383]{FT}

\subsection{The Mughal empire}

The \emph{Mughal Empire}, occupying northern India from Afghanistan to Assam and Bangladesh, and southern India almost to the coast, was founded in 1526 by Babur. It existed ceremonially up until direct british rule began in 1857.
According to Richards, the new contacts to europe contributed to the rise of a centralized power in India.\autocite[6]{richards}
\subsubsection{Early history}

Babur, a muslim, came from central Asia and entered India by first establishing himself in Kabul and pushing through the Khyber Pass.
 After a victory in Panipat in 1526, he was able to occupy much of northern India. However, the focus on military campaigns weakened the empire, and his successor Humayun was overthrown and exiled by a rebellion.\autocite[Chapter 1]{richards}

\subsubsection{Expansion and begin of decline}

Humayun was able to return in 1555, reestablishing Mughal rule.\autocite[12]{richards} Akbar, who acceded to the throne in 1556 after his fathers death, expanded the empire to encompass the entire Indian subcontinent to the north of the Godavari river. Establishing a new ruling class, a cult around himself, and increasing trade with european companies, Akbar was able to stabilize and expand the Empire.\autocite[16]{richards}

Jahangir, Akbars heir, ruled from 1605 to 1627, and was succeded by Jahan, who erected the Taj Mahal. Under Jahan, however, the financial situation of the empire was under stress due to the splendor of its court.\autocite[Chapter 6]{richards} After he fell ill, his more liberal son Dara Shikoh became regent, but was overthrown by orthodox muslims and his younger brother Aurangzeb. Even after Jahan recovered, he was declared unfit to rule and imprisoned.\autocite[Chapter 7]{richards}

\subsubsection{Aurangzeb}

Aurangzeb ruled from 1658 to 1707\autocite[165]{richards}. His legacy is controversial, with religious orthodoxy and the establishment of sharia on the one hand, but rising political and economic fortunes on the other. He was able to expand the empire to include most of southern Asia.\autocite[Chapter 8]{richards}

\subsubsection{Decline}

After Aurangzebs sons death in 1712, the dynasty was embroiled in infighting and sank in chaos. In 1719 alone, four emperors ruled succesively.\autocite[Chapter 12]{richards} How the empire fell so rapidly is debated, with Richards arguing that it was caused by the empires own prosperity allowing the provinces to seek out independence.\autocite{MughalFinances}

Under Muhammad Shah, who was able to rule until his death in 1748, the Mughal empire began to fully collapse. Thereafter, Mughal power was limited at best, with many de facto independent princes still ceremoniously paying homage to the emperor.\autocite{bose}

\subsection{The East India Company}

The East India Company began as a merchant trading house and was only incorporated as a permanent stock company in 1657.\autocite{EIC2} Up until the latter half of the 18th century, the company was not engaged in political processes.

Due to this research papers focus on india, only the East India Company will be discussed in depth.

\subsubsection{Founding}

Limited first colonial attempts, mostly focused on spice-producing islands, were soon eclipsed by private enterprises. Groups of merchants divided the significant risks, and the profits, of an expedition to eastern Asia among themselves. Soon, these temporary ventures were officialy sanctioned by the government and became stock-issuing companies in the modern sense.

Two of them, the english East India Company and the dutch United East Indian Company (the VOC, \textit{Vereenigde Oostindische Compagnie}), emerged as the main powers in the spice islands.\autocite{VOC} In the middle of the 17th century, the East India Company changed its focus to India.\autocite[Part I]{EIC}

\subsubsection{First posts in India}

Usually, bullion was exported from Britain and exchanged in India for goods.\autocite[Section \textit{The commodity structure of trade}]{FT} After focusing on the spice islands in the first decades of the 17th century, the company established two trading posts, one of whom became Chennai, and another one in Bombay. These posts, or factories, served as bases for trading with indian merchants and negotiations with the local authorities.\autocite{EIC}

\subsubsection{Trade with India}

According to Chaudhuri, ``It is inconcievable that European trade with India [...] could have been sustained on a large scale without the discovery of American silver-mines.''\autocite[395]{FT} This illustrades the trade balance between India and Europe, a problem resurfacing later with China and the tea trade. In 1621, the East India Company exported \pounds{}200,000 in bullion and coin.\autocite[398]{FT}

In 1613, the East India company first imported \textit{calico}, an indian cotton cloth, to Europe for sale. While samples had reached Europe earlier, the new bulk transport avaliable with larger ships made the import of these finished goods economically viable.\autocite[400]{FT, Calico}

While mercantilists condemned the use of indian textiles in europe, pepper imports peaked in the 1670s\autocite[399]{FT} and textile became the most important export of India. This benefitted the East India Company, having already been driven out of the spice trade (less so in pepper) by their dutch competitors.\autocite[401--407]{FT}

\subsubsection{Competitors}

Other european trading companies, namely the spanish, dutch, danish, and french, were able to establish trading posts in India. For example, the Netherlands controlled the coastal areas of Sri Lanka, having won them from the portuguese at the behest of the king of Kandy.\autocite[Chapter 5]{dutch}

The East India Company was able to expand its advantage and achieve dominance in India with the help of the English, and later British, navy. This naval power was supplanted by a private army consisting of Indian and British troops, and skilled diplomacy.\autocite{battles} The french, the last remaining competitors, were expelled during the seven years war and never regained a stable foothold.\autocite[Chapter 7]{battles} 

The portuguese were able to retain a small foothold on the western coast of India, centered around Goa. Portuguese India ceased to exist only in 1961, more than three hundred years after its inception.\autocite{PortugueseIndia}

\section{India as a colony}

\subsection{Motives and opportunities for colonization}

The territorial expansions were not motivated by strategic supremacy over european rivals, but mostly by profit. After the seven years war, the East India Company was able to aquire 

This development was enabled by the political and cultural disunity in India, and the growing technological advantage of the British over the Indians. Furthermore, many local rulers were willing to accept British influence for economic and political advantages.

\subsection{British military conquests}

\subsubsection{Bengal}
The first major military action by the Company was the conquest of Bengal, an area shared today between India and Bangladesh. In 1717, the Company had been granted a\textit{firman} by the Mughal emperor, freeing them from all inland duties in Bengal.\autocite[257]{RF} Siraj-ud-Daula, ruler of Bengal since 1756, wantet to reverse this grant and attempted to force the British out by force. A year later, this led to open war.\autocite[258]{RF}

In the battle of Plassey, the last independent \textit{navab} of Bengal and his french allies were defeated.\footnote{This conflicts was a part of the seven years war.} The british, commanded by colones Clives, installed a puppet \textit{navab}, Mir Jafar. He acceded to all british demands, and from now of Bengal was de facto a british colony.\autocite[268]{RF}

\subsubsection{Other Indian rulers}

After these first colonial conquests, smaller rulers were willing to accept british rule\autocite[Chapter 4, sections 8 and 9]{RF}, and the Marathas remained as the most powerful Indian empire. The British had made arrangements with local rulers to act as a bulwark against Magratha invasion.\autocite[273]{RF}

Soon, internal differences gave the British an opportunity for action against the Magh\-ratas, and in 1803, lord Lake entered Dehli. By the second decade of the nineteenth century, the British had conclusively defeated any opposition in northern India.\autocite[274]{RF}

Expanding British supremacy over central India was not difficult, as local lords were bankrupt and not able to resist british demands. They lost their military power, but remained as nominal rulers recieving pensions from the British.\autocite[276]{RF}

\subsubsection{Serious challenges}

Some areas groups proved more resistant to colonialization. Only after the first Anglo-Burmese war 1824--1826, the areas of Assam and Manipur became protectorates. Punjab in the north was annexed after the long and bloody Anglo-Sinkh wars between 1848 and 1852.\autocite[277]{RF}

\subsection{Tea}

By the 18th century, tea had become a popular export of China to Europe. However, the chinese were not interested in any indian or european product, so they had to be paid in silver. This depleted the silver reserves of Great Britain, and by extension of the whole of Europe.\autocite{Tea}

In response, merchants tried to export opium, a popular and illegal drug in China. Soon, however, chinese officials cracked down on the smuggling, leading to the Opium Wars.\autocite{Opium} With the market open, poppys were grown in India, made into Opium, and traded for tea.\autocite{Tea}

As an alternative, expeditions were sent to attempt to grow tea in India, grounded in fears of a Chinese trade embargo similar to the one in Japan. Tea seeds were taken from China, and skilled tea growers were enticed, but often prevented from leaving by the chinese authorities.\autocite{Tea}

In the end, tea was dicovered in Assam in 1821, and the fist shipment of tea from India reached England in 1838. This matter was of so great importance that a committee of twelve was formed specifically tasked with cultivating tea in India. Today, there are 41,000 tea gardens in Assam, covering about 26 million acres.\autocite{roy}

\section{Analysis}
\section{Conclusion}
\clearpage
\appendix
\printbibliography

\end{document}