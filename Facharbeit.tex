\documentclass[11pt, a4paper]{scrreprt}
\usepackage[T1]{fontenc}
\usepackage[utf8]{inputenc}
\usepackage{geometry}
\usepackage[style=footnote-dw, backend=biber]{biblatex}
\usepackage{url}
\bibliography{Facharbeit}
\begin{document}
\renewcommand{\baselinestretch}{1.5}
\subject{Facharbeit}
\author{Jakob Sinn}
\title{Influence of the trade between \\ Asia and Europe on India}
\maketitle
\tableofcontents
\pagenumbering{Roman}
\chapter{Introduction}
\pagenumbering{arabic}
\section{Summary}

\section{Thesis Statement}
My thesis stament: the trade between Asia and Europe, in both directions, had a profound effect on India by introducing Islam, influencing the political structure, and forming the basis for colonisation.
\section{Methodology}
 
\chapter{Trade up to the age of Islam}
Trade in this time, before the trade movements were consolidated under the peace and security offered by the Islamic conquests of Arabia, moved both on the land route via the middle east and from India over the india ocean to the Red Sea, and from there on to Egypt or the eastern coast of the Medditeranean. For both of these routes, the Medditeranean served as the distributor to Europe.

\section{Traded goods}
While most of the goods traded were spice and other special, often medical, plants, luxuries like silk and gems were also increasingly in demand from a prosperous upper class in the roman empire.
\subsection{Spice}
At the time, spices were appreciated not just as enhancers of food, but primarily as medicine.\cite{MST} With more and more trade in the Medditeranenan in general, the need for these raw material to be used in drug production rose dramatically.
The spices came from India, modern-day Indonesia, and China. Cinnamon, for example, is native to Sri Lanka, Myanmar, Banglasesh, and the Malabar Coast of India itself\cite{Cinnamon}.
\subsection{Precious stones}
Gems were and are easy to transport, do not decay, and in demand wherever an elite wants to mark its status. Thus, they were often transoprted along with the other goods on the routes.

In this era, the stones were often engraved, a technique found in almost all early cultures. The craftsmen performing this work in Greece and Rome were so famous that many of thier names have surivied to this day. Precious stones were given as state gifts and sacrificed to the gods.

\subsubsection{Cultural impact in Europe}

While they were often used as payment, these engraved stones could also be used as seals. An example are the \emph{Gemma Augustea}, with whom orders could be signed by the emperors closest associates. The name of the artisan, \emph{Dioskurides}, has survived, showing that these artists were valued along with their products.
\section{Trade routes}
In this age, India was mostly a distribution hub for the spices and other luxury goods from Indonesia and China. Reports of trading fleets exchanging goods in Indian harbours even made it all the way to ancient Greece and Rome, where they were written down and are still preserved today, for example in the writings of Ptolomy.

\subsection{To India from the southwest}

Spices were mainly originating from the islands of Indonesia and Malaya, and, over their history, many empires and kingdoms thrived of controlling the trade routes between the isles.

The stories of their \emph{Raft-men}, who came to India on rudderless rafts to sell the spices and return home with traded goods, traveled along the trading routes to Europe. They are mentioned, for example, by Pliny.

\subsection{From China}

From China, there were three routes.

The northernmost, known as the \emph{Scythian route}, went from the north of the yellow river, next to the central silk road. In modern-day northwest China, it seperated, and made its way north of the Aral and Caspian Sea, passing through Volgograd and ending on the mouth of the river Don on the Black sea, the ancient Greek colony of Tana\"{i}s. This, of course, opened the way to Byzantium via the Black Sea.

Most of the non-spices exported from China, primarily silk, were carried along the \emph{Silk Road}. The route tended to start in northern China, bypassing Tibet, the Himalayas, ans thus India, to the north. From there, it led through Persia to Medditeranen ports like Antioch.

Many spices were imported into the southern ports of China like Canton, from where spices and other goods were also exported by sea to modern-day Sri Lanka. Furthermore, goods were also shipped overland from southern China to India. The importance of this route must not be underastimated; Chinese records describe Roman trade missions in cities in modern-day northern Vietnam, and Ptolemy tells of a place called \emph{Cattigara}, identified today as \emph{\'{O}c Eo}, an archeological site in the Mekong River delta where Roman gods from the second century CE have been excavated.

\subsection{India as a hub}
India, while producing many of the goods traded, also served as a trading centre and main hub of the Asia-Europe trade. This was due to its geographical advantages with acces to Arabia and the Red Sea to the west, as well as th Indian ocean and land connections to China to the east.

In general, trade flowed mostly eastwards, while the recipients in Europe mostly paid in gold coin. Indias position as a trading centre, especially after the emergence of the Roman empire, is demontrated by find of Roman-struck gold coinage from the imperial age.

\subsection{From India to Europe}


\subsection{Development under the Romans}

During the Roman empire, both diplomacy and war opened up new routes. Ships from Europe even made it to southern India, bringing home a new Spice: black pepper. In addition, the Romans were also willing to pay for spices to add them to their meals, making spice trade even more important in relation to other imports.
\chapter{Networks of Trade up to the first European Outposts}
\chapter{Trading under colonial rule and up to Indian independence}
\chapter{Effects on India}
\chapter{Conclusion}
\appendix
\printbibliography

\end{document}